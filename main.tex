\documentclass[twoside,spanish,11pt]{book}

\usepackage[utf8]{inputenc}
\usepackage[T1]{fontenc}
\usepackage[spanish,es-tabla,english]{babel}

\usepackage[usenames, dvipsnames]{xcolor}

\usepackage{array}
\usepackage{subcaption}
\usepackage[toc,page]{appendix}
\usepackage{multirow}
\usepackage{indentfirst}
\usepackage{cite}

\usepackage{ulem}
\renewcommand{\uline}[1]{\textit{#1}}
\definecolor{MyOrange}{rgb}{0.9,0.6,0.0}
\definecolor{MyDarkBlue}{rgb}{0,0,0.8}
\definecolor{MyDarkRed}{rgb}{0.6,0,0.0}
\newcommand{\borrar}[1]{\textcolor{MyDarkRed}{\sout{#1}}} %\sout}
\newcommand{\nuevo}[1]{\textcolor{MyDarkBlue}{\uline{#1}}} %\sout}
\newcommand{\nota}[1]{\textcolor{MyOrange}{#1}} %\sout}

\reversemarginpar
\usepackage{setspace}
\usepackage{todonotes}
\newcounter{mycomment}
\newcommand{\mycomment}[3][]{%
% initials of the author (optional) + note in the margin
    \refstepcounter{mycomment}%
    {%
        \setstretch{0.7}% spacing
        \todo[color=#3,size=\scriptsize]{%
            \tiny{\textbf{(\themycomment)[\uppercase{#1}]:}~#2}}%
    }}
% For notes, corrections and suggestions
\definecolor{GRCOLOR}{rgb}{1,0.2,0.5}
\definecolor{ODRCOLOR}{rgb}{0.2,0.75,0.25}
\definecolor{LRCOLOR}{rgb}{0.2,0.3,1}
\definecolor{MyDarkRed}{rgb}{0.6,0,0.0}

\newcommand{\GR}[1] { \mycomment[GR]{#1}{GRCOLOR}}
\newcommand{\ODR}[1]{ \mycomment[ODR]{#1}{ODRCOLOR}}
\newcommand{\LR}[1]{ \mycomment[LR]{#1}{LRCOLOR}}

\newcommand{\newGR}[1]{\textcolor{GRCOLOR}{\uline{#1}}} %\sout}
\newcommand{\delGR}[1]{\textcolor{GRCOLOR}{\sout{#1}}} %\sout}
\newcommand{\newODR}[1]{\textcolor{ODRCOLOR}{\uline{#1}}} %\sout}
\newcommand{\delODR}[1]{\textcolor{ODRCOLOR}{\sout{#1}}} %\sout}
\newcommand{\newLR}[1]{\textcolor{LRCOLOR}{\uline{#1}}} %\sout}
\newcommand{\delLR}[1]{\textcolor{LRCOLOR}{\sout{#1}}} %\sout}
\newcommand{\note}[1]{\textcolor{MyDarkRed}{#1}} %\sout}
\newcommand{\deletenote}[1]{\textcolor{MyDarkRed}{\sout{#1}}} %\sout}


%\newcommand{\delsg}[1]{\textcolor{SGCOLOR}{\sout{#1}}} %\sout}

\usepackage{graphicx}
\usepackage{listings}
\usepackage{float}

\usepackage{rotating}

\usepackage[hyphens]{url} % Hay que cargarlo antes que hyperref para URLs largas
\usepackage[hidelinks=true]{hyperref}
\hypersetup{
%	colorlinks,
    linkcolor={blue!80!black},
    citecolor={blue!80!black},
    urlcolor={blue!80!black}
}
%\hypersetup{
%    colorlinks,
%    linkcolor={black},
%    citecolor={black},
%    urlcolor={black}
%}

\usepackage{marginnote}
\renewcommand*{\marginfont}{\scriptsize\color{red}\sffamily}

\usepackage{enumitem}
\setitemize{noitemsep,topsep=0pt,parsep=0pt,partopsep=0pt,itemsep=2mm}%
\setenumerate{noitemsep,topsep=0pt,parsep=0pt,partopsep=0pt,itemsep=2mm}
% \setlist{noitemsep}
%% \setlist[itemize]{itemsep=0.5mm}
%% \setlist[enumerate]{itemsep=0.5mm}

\usepackage[textfont={footnotesize,sf},labelfont={footnotesize,sf,bf}]{caption}

\renewcommand{\labelitemi}{$\bullet$}
\renewcommand{\labelitemii}{$\circ$}
\renewcommand{\labelitemiii}{--}

\setlength{\parskip}{3mm}

%% \usepackage{natbib}
%% \usepackage{chapterbib}

\renewcommand{\baselinestretch}{1.05}


\setcounter{tocdepth}{2}
\setcounter{secnumdepth}{5}

\usepackage[big,sf,bf,pagestyles]{titlesec}

\usepackage{xspace}

%% \usepackage[a4paper,
%%   total={145mm,237mm},
%%   left=40mm,
%%   top=25mm,
%%   marginparwidth=2.0cm]{geometry}

\usepackage[a4paper,
    total={155mm,237mm},
    left=35mm,
    top=32mm,
    marginparwidth=3.0cm,
]{geometry}

\lstdefinestyle{java} {
    language=Java,
    morekeywords={record, var}
}

\lstdefinelanguage{Kotlin}{
    comment=[l]{//},
    commentstyle={\color{gray}\ttfamily},
    emph={filter, first, firstOrNull, forEach, lazy, map, mapNotNull, println},
    emphstyle={\color{OrangeRed}},
    identifierstyle=\color{black},
    keywords={!in, !is, abstract, actual, annotation, as, as?, break, by, catch, class, companion, const, constructor, continue, crossinline, data, delegate, do, dynamic, else, enum, expect, external, false, field, file, final, finally, for, fun, get, if, import, in, infix, init, inline, inner, interface, internal, is, lateinit, noinline, null, object, open, operator, out, override, package, param, private, property, protected, public, receiveris, reified, return, return@, sealed, set, setparam, super, suspend, tailrec, this, throw, true, try, typealias, typeof, val, var, vararg, when, where, while},
    keywordstyle={\color{NavyBlue}\bfseries},
    morecomment=[s]{/*}{*/},
    morestring=[b]",
    morestring=[s]{"""*}{*"""},
    ndkeywords={@Listener, @Deprecated, @JvmField, @JvmName, @JvmOverloads, @JvmStatic, @JvmSynthetic, Array, Byte, Double, Float, Int, Integer, Iterable, Long, Runnable, Short, String, Any, Unit, Nothing},
    ndkeywordstyle={\color{BurntOrange}\bfseries},
    sensitive=true,
    stringstyle={\color{ForestGreen}\ttfamily},
}

\usepackage{multirow}

\usepackage{fancyhdr}
\pagestyle{fancy}

\fancyhead[LE]{}
\fancyhead[RO]{}
\fancyhead[LO]{\sf{\footnotesize\leftmark}}
\fancyhead[RE]{\sf{\footnotesize\rightmark}}

%\fancyhead[LE,RO]{\scriptsize\textbf\elautor}
%\fancyhead[RE,LO]{\scriptsize\textbf\leftmark}

%\renewcommand{\chaptermark}[1]{
%	\markboth{\chaptername
%		\ \thechapter.\ #1}{}}

\renewcommand{\sectionmark}[1]{\markright{\thesection.\ #1}}


\setlength{\headwidth}{\textwidth}

\usepackage{lmodern,textcomp}

\usepackage{forest}
\newcommand{\basictree}[1]{\begin{forest}
                               for tree={
                                   font=\ttfamily,
                                   grow'=0,
                                   child anchor=west,
                                   parent anchor=south,
                                   anchor=west,
                                   calign=first,
                                   edge path={
                                       \noexpand\path [draw, \forestoption{edge}]
                                       (!u.south west) +(7.5pt,0) |- node[fill,inner sep=1.25pt] {} (.child anchor)\forestoption{edge label};
                                   },
                                   before typesetting nodes={
                                       if n=1
                                           {insert before={[,phantom]}}
                                           {}
                                   },
                                   fit=band,
                                   before computing xy={l=15pt},
                               }
                               #1
\end{forest}}

\title{Trabajo de Fin de Grado}
\begin{document}

    \include{chapters/front}

    \pagestyle{empty}

    \pagenumbering{gobble}

    \frontmatter

    \pagestyle{plain}

    \chapter{Resumen} \label{ch:resumen}

Este Trabajo Fin de Grado contempla el diseño y
la implementación de un sistema de
componentes para el entorno de desarrollo \textit{JAMS}, dando la
posibilidad a otros desarrolladores de expandir las capacidades de la
aplicación, añadiendo o modificando sus funcionalidades.
Para demostrar las capacidades que aporta este enfoque
basado en componentes, se ha desarrollado \textit{NES4JAMS},
que incorpora un entorno de desarrollo para la creación
de videojuegos para la consola \textit{Nintendo Entertainment System}.

    \clearpage{\pagestyle{empty}\cleardoublepage}

    \selectlanguage{english}

    \chapter{Abstract} \label{ch:abstract}

This Degree contemplates the design and development of \textit{JAMS},
a modern, lightweight and highly modifiable integrated
development environment for assembly languages.
The application comes bundled with an editor, an assembler and
a simulator for the \textit{MIPS32} architecture, having
an academic approach.
To achieve this goal, the application contains several tools that
allow the user to see the state of the project from different
perspectives. \\
This application intends to contribute to a higher-quality education
for students that are taking a course related to architectures,
processing units or assembly language.
    \clearpage{\pagestyle{empty}\cleardoublepage}

    \selectlanguage{spanish}


%	\listoftodos

    \setcounter{tocdepth}{3}
    \tableofcontents
    \clearpage{\pagestyle{empty}\cleardoublepage}

    \listoffigures
    \clearpage{\pagestyle{empty}\cleardoublepage}

    \mainmatter

    \pagestyle{fancy}

    \clearpage{\pagestyle{empty}\cleardoublepage}

    \chapter{Descripción del problema} \label{ch:descripcion-del-problema}


\section{Definición de conceptos}\label{sec:definicion-de-conceptos}

\subsection{La arquitectura de la consola \textit{NES}}\label{subsec:la-arquitectura-mips}

La \textit{Nintendo Entertainment System}, comúnmente conocida como
la \textbf{\textit{NES}}, es la segunda consola de sobremesa desarrollada por
\textit{Nintendo}.
La \textit{NES} presenta una arquitectura muy interesante:
la \textbf{unidad central de proceso (\textit{CPU})} es un procesador
de 8 bits basado en el famoso procesador \textbf{MOS 6502}\cite{MOS6502}.
La \textbf{unidad de procesamiento de imágenes (\textit{PPU})} está
diseñada para utilizar la menor memoria posible, utilizando para
el dibujado un conjunto de paletas de cuatro colores y un conjunto
de teselas.
La \textbf{unidad de procesamiento de audio (\textit{APU})} presenta
cuatro canales de sonido principales: dos de pulso, uno triangular
y uno de ruido.

\noindent Gracias a su funcionamiento mediante cartuchos y que la
\textit{NES} permitía hacer operaciones se escritura en el cartucho,
los desarrolladores podían aumentar las capacidades de la \textit{NES} de manera sencilla,
añadiendo en el cartucho más memoria, chips de procesamiento específicos
o canales de sonido.
Estos chips reciben el nombre de \textit{\textbf{mapeadores}}.\cite{MAPPERS}.

\subsection{Simuladores y emuladores}
\label{subsec:simuladores-y-emuladores}

Tanto los simuladores como los emuladores pueden definirse como un
programa de computador que imita el funcionamiento de uno o varios
componentes \textit{hardware}.
Un emulador o simulador puede imitar una arquitectura, permitiendo
ejecutar aplicaciones ajenas a la arquitectura del computador principal.

Es importante diferenciar los conceptos de \textbf{simulación}
y \textbf{emulación} a la hora de crear un programa que ejecute
código de arquitecturas externas.
La diferencia entre los simuladores y emuladores se manifiesta
en la manera en la que se implementan.

\noindent Un emulador tiene como objetivo \textbf{imitar el resultado}
que la arquitectura imitada produce al ejecutar un programa, sin tener
en cuenta el proceso interno que produce dicho resultado.
Los emuladores tienden a ser rápidos, intentando producir el resultado
de la manera más rápida y fiel posible.

\noindent Los simuladores tienen como objetivo \textbf{imitar el proceso}
que produce el resultado, simulando todos los componentes de la arquitectura.
Los simuladores tienden a ser más lentos que los emuladores, pero son más
adecuados para desarrollar y depurar aplicaciones para la arquitectura
imitada sin tener el \textit{hardware} físicamente.


\section{Descripción del problema}\label{sec:descripcion-del-problema}

Como se verá más adelante, los simuladores de la arquitectura \textit{MIPS} tuvieron
un auge importante a principios del milenio, con herramientas como \textit{MARS}\cite{MARS}
o \textit{EduMIPS64}\cite{EDUMIPS64}.
Estas herramientas estaban centradas principalmente en el \textbf{ámbito educativo}, proporcionando
interfaces sencillas y pocas herramientas.
Con el paso de los años, estas herramientas han quedado obsoletas, impidiendo desarrollar aplicaciones
en ensamblador \textit{MIPS32} de una manera cómoda.

\noindent Los principales problemas que presentan estas aplicaciones son los siguientes:
\begin{itemize}
    \item \textbf{Falta de herramientas}: los entornos de desarrollo \textit{MIPS} sufren de una
    carencia grave de herramientas importantes.
    Este problema afecta principalmente a programas como \textit{EduMIPS64}, donde no existe un editor de texto y
    el ensamblador es una aplicación separada que require la línea de comandos para funcionar.
    \item \textbf{Falta de estructura de proyecto}: ninguna aplicación actual presenta una estructura
    de proyecto, indispensable para el desarrollo de aplicaciones medianas y grandes.
    \item \textbf{Falta de personalización}: al ser aplicaciones muy sencillas, ninguna permite
    personalizar la apariencia de la interfaz.
    \item \textbf{Obsolescencia}: las aplicaciones están desarrolladas con tecnologías antiguas,
    con interfaces similares a las de los programas de principio de siglo.
    \item \textbf{Falta de capacidad de expansión}: el problema más grave que presentan estas aplicaciones
    es la nula capacidad de expansión de sus características, impidiendo que otros desarrolladores
    añadan nuevas funcionalidades de manera sencilla.
\end{itemize}


\section{Objetivos}\label{sec:objetivos}

El objetivo principal de este proyecto es el desarrollo de \textbf{\textit{JAMS}
(\textit{Just Another MIPS Simulator})}, un \textbf{entorno de desarrollo integrado ligero, expandible y moderno}
centrado en los lenguajes ensamblador y, más concretamente, en el lenguaje ensamblador \textit{MIPS}.

\noindent Específicamente, en este apartado se detallarán diferentes objetivos.

\subsection{Investigar sobre las tecnologías más adecuados para el desarrollo}
\label{subsec:investigar-sobre-las-tecnologias-mas-adecuados-para-el-desarrollo}

La tecnología usada en un proyecto tiene mucho peso en el resultado final.
Es por eso que se debe buscar un conjunto de tecnologías que permitan crear una aplicación
moderna, multiplataforma y expansible.

\noexpand Este objetivo puede separarse en dos pasos:
\begin{itemize}
    \item Buscar una lenguaje de programación con capacidad para crear aplicaciones modernas,
    que permita cargar código externo a voluntad y que tenga un buen rendimiento
    tanto en la ejecución como en el desarrollo.
    \item Dentro de las posibilidades del lenguaje de programación seleccionado,
    buscar un \textit{framework} de desarrollo de aplicaciones de escritorio
    que permita desarrollar aplicaciones modernas y de gran calidad.
\end{itemize}

\subsection{Crear un entorno base y un \textit{framework} que permita implementar diferentes herramientas}
\label{subsec:crear-un-entorno-base-y-un-framework-que-permita-implementar-diferentes-herramientas}

\textit{JAMS} pretende ser un entorno de desarrollo completamente modificable.
La mejor manera para conseguir esto es desarrollar una base genérica
que se pueda ajustar posteriormente al desarrollo de una tecnología en concreto.

\noindent Esta base debe poder personalizar cualquier aspecto de la aplicación,
desde poder modificar simples mensajes y parámetros hasta poder
añadir nuevas herramientas.
Estas modificaciones pueden venir de diferentes fuentes, como lo son
los paquetes de idiomas, los paquetes de temas y los complementos.

\noindent Por último, \textit{JAMS} debe permitir al usuario configurar los parámetros
de la aplicación de manera sencilla, proporcionando una interfaz de configuración
cómoda de usar.

\noindent Para conseguir este resultado, se proponen los siguientes subobjetivos:
\begin{itemize}
    \item Desarrollar una aplicación base en conjunto con una \textit{API} que pueda
    ser usada por las diferentes herramientas para tecnologías concretas.
    \item Desarrollar e investigar diferentes formatos que permitan guardar
    y modificar los diferentes elementos estáticos de la aplicación, como lo son
    los idiomas, los temas y la configuración.
\end{itemize}

\noindent Este objetivo está altamente relacionado con los
objetivos principales del Trabajo de Fin de Grado del Grado en
Diseño y Desarrollo de Videojuegos: desarrollar un \textbf{sistema de carga
dinámica de componentes}, permitiendo al usuario activar y desactivar
componentes a voluntad sin tener que reiniciar la aplicación y
desarrollar tecnologías que permitan \textbf{modificar}
todos los aspectos de \textit{JAMS} mediante componentes.
Aunque los objetivos de los dos Trabajos de Fin de Grado están
correctamente definidos, los procesos de desarrollo para lograrlos están
fuertemente vinculados.
Es por este motivo por el que se mencionará mucho la idea de \textbf{componente}
(un programa externo que modifica el comportamiento de una aplicación principal),
aunque ninguno de los objetivos de este Trabajo de Fin de Grado requiera el
desarrollo de estos.

\subsection{Desarrollar un entorno de desarrollo para la arquitectura \textit{MIPS32}}
\label{subsec:desarrollar-un-entorno-de-desarrollo-para-la-arquitectura-mips32}

Una vez se tenga una base sólida, se ha de desarrollar un editor, un ensamblador
y un simulador para la arquitectura \textit{MIPS32r6}.
Estas tres tecnologías se apoyarán en diferentes herramientas que complementarán
su utilización, como puede ser el caso del explorador en el editor y el visualizador
de memoria en el simulador.

\noindent Este objetivo se desarrollará en dos pasos:

\begin{itemize}
    \item Desarrollar un editor, un ensamblador y un simulador usando la tecnología
    proporcionada por la base.
    Estos tres componentes deben ser expandibles mediante componentes externos.
    \item Desarrollar diferentes herramientas que complementen las funcionalidades
    del editor, del ensamblador y del simulador.
\end{itemize}

\noindent Cabe destacar que no es un objetivo el permitir crear código válido para
entornos \textit{MIPS} reales.

\section{Metodología}\label{sec:metodologia}

Debido a que la aplicación está dividida en tres secciones (tecnologías, base y entorno de desarrollo \textit{MIPS}),
es muy importante marcar una metodología que permita un desarrollo consistente, eficiente y rápido.
A nivel de desarrollo se ha optado por una metodología ágil, con \textit{sprints} de varios meses
donde se desarrollan una serie de características claves.
Todas las características se someten a varias iteraciones donde se modifican y mejoran hasta lograr un
buen resultado.

\noindent También se ha seguido un desarrollo basado en pruebas unitarias,
permitiendo mantener la calidad del código mientras la aplicación va evolucionando.

\noindent Toda la metodología ha sido implementada mediante las herramientas proporcionadas por \textit{GitHub}.
Los estados de las características asignadas a un \textit{sprint} son documentados mediante proyectos.
Las acciones obligan a que las pruebas unitarias deban superarse sin errores si se desea añadir un nueva
característica a la rama principal.
Estas acciones también se emplean para generar los binarios cuando un \textit{sprint} es superado y una
nueva versión de \textit{JAMS} es lanzada.

%\begin{figure}[H]
%    \centering
%    \includegraphics[width=\textwidth]{images/introduction/github}
%    \caption{Proyecto de \textit{GitHub} para la versión 0.4-BETA}
%    \label{fig:introduccion-github}
%\end{figure}
    \clearpage{\pagestyle{empty}\cleardoublepage}

    \chapter{Antecedentes}\label{ch:antecedentes}


\section{Antecedentes en los sistemas de componentes}
\label{sec:antecedentes-en-los-sistemas-de-componentes}

Los componentes son una parte \textbf{importante} de muchas aplicaciones
orientadas a los usuarios finales.
En esta sección se presentarán algunas aplicaciones que permiten
la instalación de componentes:

\begin{itemize}
    \item \textbf{Navegadores web}: los navegadores web
    son las principales aplicaciones con complementos.
    Programas como \textit{Firefox}\cite{FIREFOX} o
    \textit{Google Chrome}\cite{CHROME}
    presentan una \textbf{tienda} donde el usuario puede instalar
    componentes con un solo clic.
    Estos componentes suelen estar desarrollados en \textit{JavaScript}
    o \textit{TypeScript}.
    \item \textbf{Entornos de desarrollo integrados}: los entornos de
    desarrollo más conocidos también tienen la capacidad de expandir
    sus herramientas mediante componentes.
    Estos componentes pueden introducir desde pequeños cambios
    a \textbf{tecnologías completamente nuevas} a la aplicación, y suelen
    estar programados en el mismo lenguaje que el entorno de desarrollo.
    Ejemplos de entornos de desarrollo con componentes
    son \textit{IntelliJ IDEA}\cite{INTELLIJIDEA} y \textit{Eclipse}\cite{ECLIPSE},
    ambos con componentes desarrollados en \textit{Java}\cite{JAVA}.
    \item \textbf{Videojuegos}: existe una gran variedad
    de videojuegos con posibilidad de expansión mediante
    componentes, llamados \textit{\textbf{mods}} en este entorno.
    Las librerías que proporcionan soporte para \textit{mods}
    pueden estar desarrolladas por los jugadores o venir
    integradas en el juego base.
    Un ejemplo de librerías desarrolladas por jugadores es
    \textit{Spigot}\cite{SPIGOT} para \textit{Minecraft}, y un ejemplo
    de librerías integradas por los desarrolladores
    serían los videojuegos de \textit{Steam} con soporte
    para \textit{Steam Workshop}\cite{STEAM_WORKSHOP}.
    \item \textbf{Aplicaciones en la nube}: existen
    algunas aplicaciones en la nube que permiten
    al usuario complementar su experiencia con diversos
    componentes desarrollados por terceros.
    Es el caso de \textit{Google Drive}, el cual permite
    instalar componentes para la visualización y
    edición de archivos.
    Este enfoque de los componentes es muy interesante,
    ya que es el \textbf{proveedor de la aplicación} el que
    instala el componente y no el usuario.
\end{itemize}


\section{Antecedentes en el desarrollo para la consola \textit{NES}}
\label{sec:antecedentes-en-el-desarrollo-para-la-consola-nes}

Existe una gran cantidad de herramientas para poder desarrollar
una aplicación o videojuego para la consola \textit{NES}\cite{NES}.
Las principales son las \textbf{proporcionadas por \textit{Nintendo}}
en la década de los 80 a las desarrolladoras, pero estas
aplicaciones son muy antiguas y necesitan un kit de desarrollo
completo que solo \textit{Nintendo} puede proporcionar.
Las aplicaciones que se utilizan actualmente están
\textbf{desarrolladas por personas externas} a la compañía,
manteniendo vivo el desarrollo y uso de esta consola.

Estas herramientas resuelven problemas muy concretos
presentes en el desarrollo de videojuegos, y ninguna
de ellas puede considerarse un entorno de desarrollo integrado.
Algunas de estas herramientas son las siguientes:

\begin{itemize}
    \item \textbf{Nestopia}\cite{NESTOPIA}:
    \textit{Nestopia} es uno de los emuladores
    para la consola \textit{NES} más importantes.
    Desarrollado en \textit{C++}, \textit{Nestopia} tiene soporte
    para más de 200 controladores de memoria y gran cantidad de periféricos.
    \emph{Nestopia} está considerado uno de los emuladores más
    fieles que existen.
    \item \textbf{Asm6f}\cite{ASM6F}:
    ensamblador de código abierto
    para el código ensamblador 6052.
    \textit{Asm6f} permite ensamblar el código en
    \textbf{archivos \textit{INES 2.0}}
    que pueden utilizarse en emuladores, además
    de tener características especiales como soporte
    para instrucciones no oficiales.
    \item \textbf{NEXXT}: es una pequeña
    aplicacion que permite editar los gráficos de los
    videojuegos de la \textit{NES}.
    La aplicación cuenta con funciones para recrear escenas
    mediante los \textit{tilesets} creados.
\end{itemize}

En el desarrollo para la consola \textit{NES} también se utiliza
una serie de formatos de archivos comunes:

\begin{itemize}
    \item \textbf{Archivos \textit{PCX}}: \textit{PCX} es un formato de imagen desarrollado
    en el año 1985 que cuenta con una codificación \textit{run-length}.
    Actualmente, el formato ha caído en desuso, pero sus propiedades
    lo hacen un \textbf{gran candidato} para almacenar gráficos
    de la \textit{NES} antes de ser ensamblados.
    Esto se debe a su capacidad de codificar los colores mediante
    una \textit{paleta} de una manera sencilla.
    Al almacenar los valores de los píxeles como índices y no
    como colores, el formato \textit{PCX} se convierte en un
    candidato ideal para gráficos dependientes de una paleta
    externa, como es el caso de los gráficos de una \textit{NES}.
    \item \textbf{Archivos \textit{iNES}}: \textit{iNES}
    es el formato de archivo más utilizado
    para almacenar videojuegos para la consola \textit{NES}.
    Este formato comienza con una cabecera con los datos que suelen
    estar codificados en el propio cartucho, como puede ser el modo
    espejo, el controlador de memoria o la región.
    La cabecera también contiene el tamaño de la región del
    programa (\textit{PGR}) y la región de gráficos (\textit{CHR})
    que van detrás de ella.
\end{itemize}
    \clearpage{\pagestyle{empty}\cleardoublepage}

    \chapter{Tecnologías relacionadas con los componentes}\label{ch:tecnologias-relacionadas-con-los-componentes}

En este capítulo se abordará la creación de un sistema de
vinculación de componentes para \textit{JAMS}, empezando por
la estructura de un componente y terminando por la carga
del mismo dentro de \textit{JAMS}.

También se abordará el desarrollo de las
diferentes tecnologías que permiten a \textit{JAMS}
tener una estructura modular, haciendo que los
componentes puedan alterarla para incluir o modificar
diferentes características.
Cabe destacar que estas tecnologías no solo son usadas
por los componentes, sino que \textbf{el propio \textit{JAMS}}
aprovecha sus virtudes para proporcionar características
dentro del programa principal.

\section{Sistema de vinculación de componentes}\label{sec:sistema-de-vinculacion-de-componentes}

\subsection{Estructura de un componente}\label{subsec:estructura-de-un-componente}

Los componentes, al igual que en muchas aplicaciones \textit{Java},
están conformados por un archivo \textit{jar} con el código que
se desea usar en la aplicación principal.
Más concretamente, un componente de \textit{JAMS} debe contener
dos elementos esenciales:
\begin{itemize}
    \item \textbf{Punto de entrada}: está conformado por una
    clase que extiende a $Plugin$, una clase proporcionada por
    \textit{JAMS} que representa un componente.
    \textit{JAMS} crea una instancia de esta clase para
    poder ejecutar el código externo.
    \item \textbf{Archivo de metadatos}: confirmado por un archivo
    \textit{JSON}\cite{JSON} $plugin.json$ en la raíz del archivo \textit{jar}.
    Este archivo contiene los parámetros globales del componente,
    como pueden ser el \textbf{nombre}, la dirección del
    \textbf{punto de entrada}, la versión, los autores,
    o la descripción.
    Un ejemplo de archivo $plugin.json$ se puede observar en la
    figura \ref{fig:plugin-json}.
\end{itemize}


\begin{figure}[h]
    \centering
    \begin{lstlisting}[frame=single,label={lst:plugin-json}]
{
  "name": "NES4JAMS",
  "main": "io.github.gaeqs.nes4jams.NES4JAMS",
  "version": "0.1-ALPHA",
  "authors": [
    "Gael Rial Costas"
  ],
  "favicon": "/gui/icon/favicon.png",
  "description_node": "NES4JAMS_DESCRIPTION"
}
    \end{lstlisting}
    \caption{Ejemplo de archivo $plugin.json$}
    \label{fig:plugin-json}
\end{figure}

\subsubsection{Dependencias}\label{subsubsec:dependencias}

Un componente puede depender de otro componente.
Para realizar una inicialización correcta de los componentes
se proporcionan los parámetros $dependencies$
y $soft\_dependencies$, los cuales pueden utilizarse en
el archivo $plugin.json$.
Todos los componentes cuyo nombre esté dentro de una de estas
listas serán inicializados con anterioridad.
Si no se encuentra ningún componente que tenga el nombre
de algún valor de $dependencies$, el componente no
se inicializará, lanzando una excepción.
Cabe destacar que dos componentes no pueden conformar
una \textbf{dependencia cíclica}.

\subsubsection{Puntos de entrada}\label{subsubsec:puntos-de-entrada}

Como ya se ha mencionado, un punto de entrada está conformado
por una \textbf{clase que extiende a $Plugin$}.
El desarrollador puede extender dos métodos definidos por esta clase:
$onEnable$ y $onDisable$, que serán llamados cuando el
componente se vincula o desvincula de la aplicación principal.
Un ejemplo de punto de entrada se puede observar en la figura \ref{fig:entry-point}.

\begin{figure}[h]
    \centering
    \begin{lstlisting}[frame=single,label={lst:entry-point},language=Kotlin]
class MyPlugin : Plugin() {

    override fun onEnable() {
        println("My plugin has been enabled!")
        if (JamsApplication.isLoaded()) {
            loadApplicationData()
        }
    }

    override fun onDisable() {
        println("My plugin has been disabled!")
    }

    @Listener
    fun onApplicationLoad(event: JAMSApplicationPostInitEvent)
        = loadApplicationData()

    private fun loadApplicationData() {
        println("Now I can access the JavaFX application!")
    }

}
    \end{lstlisting}
    \caption{Ejemplo de punto de entrada de un componente desarrollado en \textit{Kotlin}}
    \label{fig:entry-point}
\end{figure}

El punto de entrada puede ser inicializado en diferentes
etapas del proyecto: el componente se cargará antes que el contexto
de \textit{JavaFX} si este ya estaba instalado en la aplicación
cuando esta se lanza.
El desarrollador debe \textbf{comprobar si el contexto se ha creado}
antes de añadir o modificar nuevos elementos.
Si el contexto aún no se ha creado, los componentes podrán
usar el evento $JAMSApplicationPostInitEvent$ para ejecutar
código cuando este se inicialice.
Este sistema de eventos se analizará en profundidad más adelante en esta memoria.


\subsection{Vinculación de un componente}\label{subsec:vinculacion-de-un-componente}

Es posible vincular un componente de dos maneras diferentes:
cuando el usuario \textbf{instala el componente} desde la aplicación
y cuando \textbf{se inicializa la aplicación principal} y el componente
está ya instalado.
La vinculación de componentes difiere en varios aspectos en estas
dos situaciones: cuando se instala el componente, \textit{JAMS}
comprueba si \textbf{todas sus dependencias fuertes están presentes}.
Si esto no se cumple, el componente no se instala.
Cuando arranca aplicación principal debe vincular
una cantidad no definida de componentes, y por ello tiene que
generar un \textbf{grafo de dependencias} antes de inicializarlos.

\subsection{Desvinculación de un componente}\label{subsec:desvinculacion-de-un-componente}

Un componente se desvincula de la aplicación principal
cuando \textbf{el usuario lo desinstala} desde
la configuración o cuando la \textbf{aplicación principal se cierra}.
De la misma manera que en la vinculación, el proceso de
desvinculación \textbf{difiere} en ambas circunstancias.
Para que un componente sea desinstalado, el usuario
debe desinstalar previamente \textbf{todos los componentes
que dependen} del componente desinstalado.
Se puede desinstalar el componente cuando no hay
ningún componente dependiente instalado.
Cuando la aplicación principal se cierra
\textbf{no se desvincula ningún componente},
sino que se llama únicamente a sus métodos $onDisable$, dejando
el proceso de desvinculación a la \textit{JVM}.

\subsection{Interfaz de usuario}\label{subsec:interfaz-de-usuario}

Los usuarios pueden instalar o desinstalar componentes
desde la \textbf{ventana de configuración}, mostrada en la figura \ref{fig:plugin-ui}.
En esta interfaz se mostrará la lista de componentes
que están instalados junto con su nombre, su versión
y su descripción.

\begin{figure}[h]
    \centering
    \includegraphics[width=\textwidth]{images/componentes/plugin-ui}
    \caption{Sección de componentes en la configuración}
    \label{fig:plugin-ui}
\end{figure}

\section{Gestores}\label{sec:gestores}

Toda la arquitectura de la aplicación está basada en \textbf{gestores}.
Un gestor se puede definir como un conjunto de elementos que las herramientas pueden utilizar.
\textit{JAMS} proporciona tres tipos básicos de gestores:
\begin{itemize}
    \item \textbf{Gestores normales}: implementados por la clase \textit{Manager}.
    Contienen una lista de elementos sin ninguna jerarquía.
    \item \textbf{Gestores con valor por defecto}: un gestor de este tipo actúa
    como un gestor normal, con la diferencia que uno de sus valores es el valor por defecto.
    Estos gestores heredan de la clase \textit{DefaultValuableManager}.
    \item \textbf{Gestores con valor seleccionado}: actúan como los gestores con valor por defecto,
    pero manteniendo seleccionado uno de los elementos.
    Cuando el elemento seleccionado se elimina, el elemento por
    defecto es el que queda seleccionado en su lugar.
    Estos gestores heredan de la clase \textit{SelectableManager}.
\end{itemize}

\subsection{Proveedores}\label{subsec:proveedores}

Cada elemento guardado en un gestor \textbf{está asociado al proveedor que lo proporciona}.
Un proveedor puede ser un componente o el propio \textit{JAMS}.
Cuando un proveedor se desvincula de la aplicación, todos
los elementos que proporciona son eliminados de los gestores.
Este diseño es una de las principales razones por las que la carga y descarga de
componentes es viable en \textit{JAMS}.
Al tener todos los elementos correctamente etiquetados, \textit{JAMS}
puede retirar los elementos de un componente cuando este se
descarga con una simple llamada a cada gestor.

\subsection{Registro}\label{subsec:registro}

El registro es un \textbf{elemento estático dentro de la aplicación}.
Se le puede considerar como un \textbf{gestor de gestores}.
En el registro se pueden recuperar, añadir, eliminar o modificar gestores.
Igual que sucede con los gestores normales, cuando un proveedor se desvincula de la aplicación,
todos los gestores que proporciona se eliminan del registro.

\textit{JAMS} permite separar los gestores en dos tipos:
\textbf{gestores primarios} y \textbf{gestores secundarios}.
Los gestores primarios son fácilmente accesibles cuando se busca un gestor por tipo
usando métodos como $Manager.of(Type.class)$.
Solo puede existir un gestor primario por tipo.
Para buscar gestores secundarios, se debe proporcionar el nombre del gestor explícitamente.

\subsection{Uso de los gestores}\label{subsec:uso-de-los-gestores}

Como ya se ha indicado, un componente puede añadir, eliminar
y modificar \textbf{los elementos de los gestores y los gestores en sí}.
Definir nuevos gestores es muy sencillo: simplemente se debe crear
una nueva clase que extienda a $Manager$ o a alguno de sus hijos,
como se puede ver en la figura \ref{fig:manager-definition}.

\begin{figure}[h]
    \centering
    \begin{lstlisting}[frame=single,label={lst:manager-definition},language=Kotlin]
data class MyElement(
    private val name: String,
    private val provider: ResourceProvider
) : ManagerResource {
    override fun getName() = name
    override fun getResourceProvider() = provider
}

class MyManager(provider: ResourceProvider) : Manager<MyElement>(
    provider,
    "my-manager",
    MyElement::class.java,
    false
) {
    override fun loadDefaultElements() {}
}
    \end{lstlisting}
    \caption{Definición de un un gestor}
    \label{fig:manager-definition}
\end{figure}

Una vez definido el gestor, este hay que \textbf{registrarlo}.
Para ello, se debe de acceder al registro.
Una vez registrado, cualquier componente podrá acceder al
gestor empleando los métodos proporcionados por el propio \textit{JAMS},
como se observa en la figura \ref{fig:manager-use}.

\begin{figure}[h]
    \centering
    \begin{lstlisting}[frame=single,label={lst:manager-use},language=Kotlin]
class MyPlugin : Plugin() {

    private fun registerAndUseManager() {
        Jams.REGISTRY.registerPrimary(MyManager(this))

        val manager = Manager.of(MyElement::class.java)
        manager.add(MyElement("Test element", this))

        manager.filter { it.name.startsWith("T") }
            .forEach { println(it.name) }
    }
}
    \end{lstlisting}
    \caption{Definición de un un gestor}
    \label{fig:manager-use}
\end{figure}


\section{Eventos}\label{sec:eventos}

\textit{JAMS} incluye un sistema de eventos que permite informar
de sucesos entre componentes de la aplicación.
Este sistema está profundamente inspirado en el sistema de eventos
utilizado por la comunidad de \textit{Minecraft}
en proyectos como \textit{Spigot}\cite{SPIGOT}
o \textit{Sponge}\cite{SPONGE}, y puede considerarse una evolución descentralizada
de esta tecnología.

\subsection{Emisores de eventos}\label{subsec:emisores-de-eventos}

Los emisores de eventos son los encargados de transmitir
eventos a los elementos que los escuchan.
Los diferentes creadores de eventos los transmitirán a través de un
canal que será utilizado también por los elementos que estén
escuchando a la espera de recibirlos.

Un emisor de eventos está representado por la interfaz $EventBroadcast$.
Esta interfaz es implementada por cualquier elemento que pueda emplearse para registrar escuchas.
Los gestores, los componentes o el propio \textit{JAMS} implementan
un emisor por defecto.
La clase $SimpleEventBroadcast$ contiene una implementación de $EventBroadcast$
que puede utilizarse como superclase.

\subsection{Definir escuchas}\label{subsec:definir-escuchas}

Las escuchas son \textbf{métodos no estáticos marcados con la anotación @Listener}.
Estos métodos solo tienen un parámetro que pide un elemento que extienda la clase
$Event$ y deben devolver $void$.
Este método, después de ser registrado en un gestor de idiomas,
se ejecutará cuando se añada un nuevo idioma.
En la figura \ref{fig:listener} se observa una escucha de ejemplo.

\begin{figure}[h]
    \centering
    \begin{lstlisting}[frame=single,label={lst:listener},language=Kotlin]
@Listener
fun onLanguageRegister(
    event: ManagerElementRegisterEvent.After<Language>
) {
    println("New language available! ${event.element.name}")
}
    \end{lstlisting}
    \caption{Objeto registrando un evento en el emisor general de \textit{JAMS}}
    \label{fig:listener}
\end{figure}

A diferencia de otros sistemas similares,
el sistema de eventos de \textit{JAMS} permite usar \textbf{eventos genéricos}.
Un ejemplo es el caso anterior, donde el método solicita un elemento de tipo
$ManagerElementRegisterEvent.After<Language>$.
Si el emisor al que está registrado emite un evento de tipo
$ManagerElementRegisterEvent.After<Theme>$, la escucha no será invocada.

Un evento puede extender la clase de otro evento.
Esto permite generar una jerarquía de eventos.
Una escucha que pide un cierto tipo de evento se ejecutará siempre
que se produzca dicho evento o uno de sus hijos.
Si una escucha \delODR{pide el}\newODR{¿invoca al?} evento $Event$,
su método se ejecutará siempre que ocurra un evento.

Algunos eventos implementan la interfaz $Cancellable$, lo cual permite cancelar el evento.
Las escuchas restantes no serán llamadas cuando se cancela un evento salvo que se defina lo contrario
en la anotación $@Listener$.

\subsection{Registrar escuchas}\label{subsec:registrar-escuchas}

Una vez un objeto tenga definidas todas sus escuchas, estas
pueden registrarse en uno o varios emisores de eventos.
Para ello se debe llamar al método $registerListeners$ del
emisor deseado, tal y como se muestra en la figura \ref{fig:event-registration}.

\begin{figure}[h]
    \centering
    \begin{lstlisting}[frame=single,label={lst:event-registration-use},language=Kotlin]
class ObjectWithListeners {

    init {
        Jams.getGeneralEventBroadcast().registerListeners(
            /* instance = */ this,
            /* useWeakReferences = */ true
        )
    }

    @Listener
    private fun onEvent(event: Event) {
        println("I received a new event! $event")
    }

}
    \end{lstlisting}
    \caption{Objeto registrando un evento en el emisor general de \textit{JAMS}}
    \label{fig:event-registration}
\end{figure}

El método de registro pide dos variables: el objeto
con las escuchas y si se deben emplear \textbf{referencias débiles}.
Esta última opción permite que los objetos sean eliminados de la memoria
cuando ya no resulten necesarios, incluso aunque tengan escuchas registradas.
Estas escuchas \textbf{se eliminarán del emisor automáticamente} si no existe
ninguna referencia al objeto (sin contar la del propio emisor).
Esto permite desarrollar componentes de una manera \textbf{muy sencilla}:
combinado con el sistema de proveedores, las referencias débiles
de las escuchas evitan que la aplicación mantenga elementos de un
componente cuando este se desinstala.

Por último, cabe destacar que el método de registro
busca todos los métodos de escucha de un objeto,
\textbf{ampliando la búsqueda a los métodos de las clases padre}.
Esto permite heredar funcionalidades de objetos ya programados
de manera sencilla.
Si el desarrollador desea registrar un único método puede hacerlo
con los métodos proporcionados, pero para ello se requieren
conocimientos de la librería \textit{reflection} de \textit{Java}.


\section{Tareas}\label{sec:tareas}

Las tareas son elementos muy importantes dentro de una aplicación
de escritorio.
Consisten en \textbf{funciones que ejecutan una tarea de manera
asíncrona al hilo principal} de la aplicación, devolviendo un
resultado al finalizar su ejecución.
\textit{Java} proporciona diversas maneras de gestionar
las tareas dentro de una aplicación, y \textit{JavaFX}
también implementa su propia solución.
Por ello \textit{JAMS} proporciona una pequeña
librería que permite a los desarrolladores ejecutar
tareas \textbf{visibles por el usuario} en la barra
de proceso de la ventana principal de la aplicación.

\textit{JAMS} implementa por defecto un
sistema de tareas basado en \textbf{ejecutor}.
Un ejecutor es un conjunto de hilos que ejecutan
diversas tareas.
Cada proyecto presenta un ejecutor único por defecto,
lo que permite los desarrolladores lanzar tareas de una
manera sencilla, como se puede observar en la figura \ref{fig:tasks-execution}.

\begin{figure}[h]
    \centering
    \begin{lstlisting}[frame=single,label={lst:tasks-execution},language=Kotlin]
fun startTask(project: Project) {
    project.taskExecutor.execute(
        LanguageTask.of("MY_TITLE_LANGUAGE_NODE") {
            Thread.sleep(1000)

            // Do things here!

            Thread.sleep(1000)
    })
}
    \end{lstlisting}
    \caption{Inicialización de una tarea con título}
    \label{fig:tasks-execution}
\end{figure}

Las tareas creadas con la clase $LanguageTask$
pueden tener un título y una descripción que cambian dependiendo
del idioma seleccionado por el usuario.
Ambos son visibles al usuario, y se muestran
en la barra inferior de la ventana principal de \textit{JAMS},
como se puede ver en la figura \ref{fig:jams-assembling}.

\begin{figure}[h]
    \centering
    \includegraphics[width=\textwidth]{images/tecnologias/jams-assembling}
    \caption{\textit{JAMS} ejecutando una tarea de ensamblaje}
    \label{fig:jams-assembling}
\end{figure}
    \clearpage{\pagestyle{empty}\cleardoublepage}

    \chapter{Tecnologías relacionadas con los componentes}\label{ch:tecnologias-relacionadas-con-los-componentes}

En este capítulo se abordará el desarrollo de las
diferentes tecnologías que permiten a \textit{JAMS}
tener una estructura modular, haciendo que los
componentes puedan modificarla para incluir o modificar
diferentes características.
Cabe destacar que estas tecnologías no solo son usadas
por los componentes, sino que \textbf{el propio \textit{JAMS}}
aprovecha sus virtudes para proporcionar características
dentro del programa principal.

\section{Gestores}\label{sec:gestores}

Toda la arquitectura de la aplicación está basada en \textbf{gestores}.
Un gestor se puede definir como un conjunto de elementos que las herramientas pueden usar.
\textit{JAMS} proporciona tres tipos básicos de gestores:
\begin{itemize}
    \item \textbf{Gestores normales}: implementados por la clase \textit{Manager}.
    Contienen una lista de elementos sin ninguna jerarquía.
    \item \textbf{Gestores con valor por defecto}: actúa como un gestor normal, con la
    diferencia que uno de sus valores es el valor por defecto.
    Estos gestores heredan de la clase \textit{DefaultValuableManager}.
    \item \textbf{Gestores con valor seleccionado}: actúan como un gestor con valor por defecto,
    pero con uno de los elementos seleccionado.
    Cuando el elemento seleccionado se elimina, el elemento por defecto queda seleccionado.
    Estos gestores heredan de la clase \textit{SelectableManager}.
\end{itemize}

\subsection{Proveedores}\label{subsec:proveedores}

Cada elemento guardado en un gestor \textbf{está asociado al proveedor que lo proporciona}.
Un proveedor puede ser un componente o el propio \textit{JAMS}.
Cuando un proveedor se desvincula de la aplicación, todos los elementos proporcionados
por el proveedor son eliminados de los gestores.
Este diseño es una de las principales razones por las que la carga y descarga de
componentes es viable en \textit{JAMS}.
Al tener todos los elementos de correctamente etiquetados, \textit{JAMS}
puede retirar los elementos de un componente cuando éste se
descarga con una simple llamada a cada gestor.

\subsection{Registro}\label{subsec:registro}

El registro es un \textbf{elemento estático dentro de la aplicación}.
Se puede considerar un \textbf{gestor de gestores}.
En el registro se pueden recuperar, añadir, eliminar o modificar gestores.
Igual que los gestores normales, cuando un proveedor se desvincula de la aplicación,
todos los gestores proporcionados por el proveedor son eliminados del registro.

\noindent \textit{JAMS} permite separar los gestores en dos tipos:
\textbf{gestores primarios} y \textbf{gestores secundarios}.
Los gestores primarios son fácilmente accesibles cuando se busca un gestor por tipo
usando métodos como $Manager.of(Type.class)$.
Solo puede existir un gestor primario por tipo.
Para buscar gestores secundarios, se debe proveer el nombre del gestor explícitamente.

\subsection{Uso de los gestores}\label{subsec:uso-de-los-gestores}

Como ya se ha presentado, un componente puede añadir, eliminar
y modificar \textbf{los elementos de los gestores y los gestores en sí}.
Definir nuevos gestores es muy sencillo: simplemente se debe crear
una nueva clase que extienda a $Manager$ o a alguno de sus hijos,
como se puede ver en la figura \ref{fig:manager-definition}.

\begin{figure}[h]
    \centering
    \begin{lstlisting}[frame=single,label={lst:manager-definition},language=Kotlin]
data class MyElement(
    private val name: String,
    private val provider: ResourceProvider
) : ManagerResource {
    override fun getName() = name
    override fun getResourceProvider() = provider
}

class MyManager(provider: ResourceProvider) : Manager<MyElement>(
    provider,
    "my-manager",
    MyElement::class.java,
    false
) {
    override fun loadDefaultElements() {}
}
    \end{lstlisting}
    \caption{Definición de un un gestor}
    \label{fig:manager-definition}
\end{figure}

\noindent Una vez definido el gestor, este se debe \textbf{registrar}.
Para ello, se debe de acceder al registro.
Una vez registrado, cualquier componente podrá acceder al
gestor empleando los métodos proporcionados por el propio \textit{JAMS},
como se puede observar en la figura \ref{fig:manager-use}.

\begin{figure}[h]
    \centering
    \begin{lstlisting}[frame=single,label={lst:manager-use},language=Kotlin]
class MyPlugin : Plugin() {

    private fun registerAndUseManager() {
        Jams.REGISTRY.registerPrimary(MyManager(this))

        val manager = Manager.of(MyElement::class.java)
        manager.add(MyElement("Test element", this))

        manager.filter { it.name.startsWith("T") }
            .forEach { println(it.name) }
    }
}
    \end{lstlisting}
    \caption{Definición de un un gestor}
    \label{fig:manager-use}
\end{figure}


\section{Eventos}\label{sec:eventos}

\textit{JAMS} incluye un sistema de eventos que permite informar de sucesos entre componentes de la aplicación.
Este sistema está profundamente inspirado en el sistema de eventos usado por la comunidad de \textit{Minecraft}
en proyectos como \textit{Spigot}\cite{SPIGOT}
o \textit{Sponge}\cite{SPONGE}, y puede considerarse una evolución descentralizada
de esta tecnología.

\subsection{Emisores de eventos}\label{subsec:emisores-de-eventos}

Los emisores de eventos son los encargados de relacionar los creadores de eventos con sus escuchadores.
Un emisor de evento está representado por la interfaz $EventBroadcast$.
Esta interfaz es implementada por cualquier elemento que quiera ser usado para registrar escuchas.
Los gestores, los componentes, o el propio \textit{JAMS} implementan un emisor por defecto.
La clase $SimpleEventBroadcast$ contiene una implementación de $EventBroadcast$
que se puede usar como superclase.

\subsection{Definir escuchas}\label{subsec:definir-escuchas}

Las escuchas son \textbf{métodos no estáticos anotados con la anotación @Listener}.
Estos métodos solo tienen un parámetro que pide un elemento que extienda la clase
$Event$ y deben devolver $void$.

\begin{figure}[h]
    \centering
    \begin{lstlisting}[language=Java,style=java,frame=single,label={lst:listeners}]
@Listener
private void onLanguageRegister(ManagerElementRegisterEvent.After<Language> event) {
    System.out.println("New language available! " + event.getElement().getName());
}
    \end{lstlisting}
    \caption{Ejemplo de una escucha}
    \label{fig:listener}
\end{figure}


\noindent Este método, después de ser registrado en un gestor de idiomas,
se ejecutará cuando un nuevo idioma sea añadido al gestor.

\noindent A diferencia de otros sistemas de eventos similares,
el sistema de eventos de \textit{JAMS} permite usar \textbf{eventos genéricos}.
Un ejemplo es el caso anterior, donde el método pide un elemento de tipo
$ManagerElementRegisterEvent.After<Language>$.
Si el emisor al que está registrado emite un evento de tipo
$ManagerElementRegisterEvent.After<Theme>$, la escucha no será invocada.

\noindent Un evento puede extender la clase de otro evento.
Esto permite generar una jerarquía de eventos.
Una escucha que pide un cierto tipo de evento se ejecutará siempre que dicho evento o uno de sus hijos ocurra.
Si una escucha pide el evento $Event$, su método se ejecutará siempre que un evento ocurra.

\noindent Algunos eventos implementan la interfaz $Cancellable$, lo cual permite cancelar el evento.
Las escuchas restantes no serán llamadas cuando un evento es cancelado salvo que se defina lo contrario
en la etiqueta $@Listener$.
    \clearpage{\pagestyle{empty}\cleardoublepage}

    \chapter{Entorno de desarrollo \textit{NES}}\label{ch:entorno-de-desarrollo-nes}

En este capítulo se abordará el desarrollo de \textit{NES4JAMS},
un componente que aporta a \textit{JAMS} un entorno de desarrollo
integrado para la creación de videojuegos de la consola
\textit{NES}.
Gracias a la arquitectura de \textit{JAMS}, este componente
aprovechará muchas de las características desarrolladas
para el entorno de desarrollo \textit{MIPS32} presente
por defecto en la aplicación.
Este entorno de desarrollo consistirá de un editor
de texto, un ensamblador y un simulador.


\section{Creación de un proyecto}\label{sec:creacion-de-un-proyecto}

\begin{figure}[h]
    \centering
    \includegraphics[width=0.8\textwidth]{images/nes/nes-project-creation}
    \caption{Creación de un proyecto \textit{NES}}
    \label{fig:nes-project-creation}
\end{figure}

\textit{NES4JAMS} añade un nuevo tipo de proyecto
que incorpora todas las herramientas para el desarrollo
de videojuegos de \textit{NES}.
Los usuarios podrán crear nuevos proyectos de este tipo
desde el menú de creación de proyectos.
Como los proyectos de \textit{NES} están destinados
a una consola muy específica, el usuario solo debe
especificar \textbf{el nombre y la localización del proyecto},
como se puede observar en la figura \ref{fig:nes-project-creation}.

Una vez creado el proyecto, \textit{JAMS} mostrará una ventana
principal muy similar a la de los proyectos \textit{MIPS32}.


\section{Editor}\label{sec:editor}

Los desarrolladores de videojuegos de \textit{NES} trabajan con
dos tipos de archivo principales: los archivos \textit{asm}
para el código y los archivos \textit{pcx} para los gráficos.
\textit{NES4JAMS} permite al usuario editar los dos tipos de archivo
de manera sencilla.

Cuando el usuario desea editar uno de estos archivos,
debe seleccionarlo en la herramienta \textbf{explorador}.
Esta herramienta muestra una representación en forma de árbol
de la estructura del proyecto.
El usuario puede expandir y contraer carpetas, así como crear,
borrar y mover archivos.
Si el usuario usa el doble clic sobre un archivo editable, este se abrirá
en el editor.
Dependiendo del tipo de archivo, el editor será diferente,
como se puede observar en la figura \ref{fig:nes-editor}.

\begin{figure}[h]
    \centering
    \includegraphics[width=0.8\textwidth]{images/nes/nes-editor}
    \caption{Editor de código y de gráficos junto con el explorador}
    \label{fig:nes-editor}
\end{figure}

El menú contextual del explorador presenta varias acciones que
pueden ejecutarse sobre las carpetas y los archivos
del proyecto.
Una de las opciones más particulares es la opción de añadir o eliminar
archivos de código o de gráficos del ensamblador.
Al ser \textit{JAMS} un entorno de desarrollo basado en \textbf{proyectos},
se ha de proporcionar una manera de incluir o excluir archivos del
videojuego resultante.
Con este simple sistema, el usuario podrá elegir qué archivos se debe ensamblar.
Los archivos a ensamblar estarán marcados en \textbf{verde} en el explorador,
y aparecerán en orden en los nodos \textbf{Archivos a ensamblar} y
\textbf{\textit{Sprites} a ensamblar}.
Cabe destacar que, como se verá más adelante, los el orden de ensamblaje
importa, por lo que esta herramienta permite ordenarlos de una manera sencilla.

Una vez el usuario abra un archivo, su editor aparecerá en la herramienta
principal de la sección: \textbf{el visualizador de archivos}.
\textit{NES4JAMS} implementa dos editores nuevos a \textit{JAMS}:
el editor de código \textit{MOS 6502} y el editor de gráficos \textit{PCX}.

\subsection{Editor de código}\label{subsec:editor-de-codigo}

El editor de código usa el sistema de indexación desarrollado
en la capa base de \textit{JAMS}, por lo que este editor también
puede considerarse un \textbf{editor de texto inteligente}:
el editor convierte el texto puro en los componentes ensamblador
representados, pudiendo así aportar ayudas al usuario.
El editor también tiene conocimiento de las referencias y el alcance de
todas las etiquetas y macros, tanto en el propio archivo a editar
como en el resto de archivos a ensamblar.
El editor también incorpora un \textbf{autocompletador}.
Esta herramienta ayuda al usuario cuando escribe código
aportando sugerencias de autocompletación, como se puede
observar en la figura \ref{fig:nes-autocompletion}.

\begin{figure}[h]
    \centering
    \includegraphics[width=0.8\textwidth]{images/nes/nes-autocompletion}
    \caption{Autocompletador ayudando al usuario}
    \label{fig:nes-autocompletion}
\end{figure}

\subsection{Editor de gráficos}\label{subsec:editor-de-graficos}

El formato \textit{PCX} es un formato de imagen desarrollado
en el año 1985 que cuenta con una codificación \textit{run-length}.
Actualmente, el formato ha caído en desuso, pero sus propiedades
lo hacen un \textbf{gran candidato} para almacenar gráficos
de la \textit{NES} antes de ser ensamblados.
Esto se debe a su capacidad de codificar los colores mediante
una \textit{paleta} de una manera sencilla.
Al almacenar los valores de los píxeles como índices y no
como colores, el formato \textit{PCX} se convierte en un
candidato ideal para gráficos dependientes de una paleta
externa, como es el caso de los gráficos de una \textit{NES}.

El editor de archivos \textit{PCX} permite modificar los
gráficos del videojuego de una manera rápida y sencilla.
Este editor está pensado exclusivamente para gráficos
de \textit{NES}, por lo que cada pixel solo puede tomar
cuatro valores diferentes.
El color final se buscará en la paleta seleccionada.
El usuario puede cambiar el color de cada valor de la
paleta utilizando el botón central del ratón sobre
la casilla a cambiar.
Por motivos de accesibilidad, el usuario también puede
cambiar el color empleando el botón principal del
ratón mientras mantiene pulsada la tecla $Ctrl$.

Cabe destacar que, aunque este editor permite editar
archivos \textit{PCX} de cualquier tamaño, la consola
leerá el archivo como si tuviera un ancho de 128 píxeles.
Esto es debido a que las tablas donde se guardan los gráficos
en la consola tienen un tamaño de 16x16 patrones
de 8x8 píxeles cada uno.
Si se ensambla un archivo \textit{PCX} con otro ancho,
lo más probable es que los gráficos se conviertan en
\textbf{ruido}.
Los archivo \textit{PCX} creados por \textit{NES4JAMS}
siempre tienen el tamaño de una tabla de patrones,
es decir, de 128x128 píxeles.

\begin{figure}[h]
    \centering
    \includegraphics[width=0.8\textwidth]{images/nes/nes-graphics-change}
    \caption{\textit{Super Mario Bros.} con los gráficos modificados}
    \label{fig:nes-graphics-change}
\end{figure}

\subsection{Archivos iNES}\label{subsec:archivos-nes}

El formato \textit{iNES} es el formato de archivo más utilizado
para almacenar videojuegos para la consola \textit{NES}.
Este formato comienza con una cabecera con los datos que suelen
estar codificados en el propio cartucho, como puede ser el modo
espejo, el controlador de memoria o la región.
Esta cabecera también contiene el tamaño de la región del
programa (\textit{PGR}) y la región de gráficos (\textit{CHR})
que prosiguen a la cabecera.

\textit{NES4JAMS} es capaz de \textbf{cargar y generar} archivos
\textit{iNES} de manera nativa.
Al abrir un archivo \textit{iNES}, \textit{NES4JAMS} mostrará
el editor mostrado en la figura \ref{fig:nes-ines-editor}.

\begin{figure}[h]
    \centering
    \includegraphics[width=0.8\textwidth]{images/nes/nes-ines-editor}
    \caption{Archivo \textit{iNES} en el editor}
    \label{fig:nes-ines-editor}
\end{figure}

En este editor el usuario podrá ejecutar una simulación
para el videojuego seleccionado, permitiéndole \textbf{modificar}
los parámetros de la cabecera del videojuego a cargar.
El usuario también tiene la opción de \textbf{exportar las tablas
de patrones} del cartucho en un archivo \textit{PCX}.
Una previsualización de estas tablas se puede observar en la parte
derecha del editor.
Por último, el editor da la opción de \textbf{guardar una copia}
del videojuego con los parámetros modificados.

\subsection{Configuraciones}\label{subsec:configuraciones}

\begin{figure}[h]
    \centering
    \includegraphics[width=0.8\textwidth]{images/nes/nes-configurations}
    \caption{Menú de configuraciones}
    \label{fig:nes-configurations}
\end{figure}

Igual que en el entorno de desarrollo para \textit{MIPS32},
el entorno de desarrollo \textit{NES} permite ensamblar
el programa con diferentes \textbf{configuraciones}.
Estas configuraciones definen los \textbf{parámetros de la cabecera}
del archivo \textit{iNES} resultante y los \textbf{bancos de memoria}
presentes en el cartucho.

Los bancos de memoria están definidos por una \textbf{dirección de inicio}
y un \textbf{tamaño} en bytes.
De manera muy similar a las directivas \textit{.text}
y \textit{.data} de \textit{MIPS32}, el desarrollador podrá
alternar entre bancos de memoria usando la directiva \textit{.bank}.

\textit{NES4JAMS} también incorpora una opción que impide a un
banco de memoria ser escrito en el cartucho.
Esta funcionalidad es muy importante para algunos videojuegos
avanzados, ya que les permite utilizar un banco de memoria como
una representación de una \textbf{memoria RAM} adicional
dentro del cartucho.


\section{Ensamblador}\label{sec:ensamblador}

El ensamblador para el lenguaje ensamblador \textit{MOS 6502}
es el ensamblador usado para ensamblar videojuegos para la \textit{NES}.
Este ensamblador soporta características avanzadas empleadas
comúnmente al programar en ensamblador, como las macros,
las etiquetas globales o las referencias relativas.
El ensamblador ensambla el código de un proyecto en \textbf{cuatro pasos}:
descubrimiento, expansión, asignación de direcciones y asignación de valores.
Estos son los mismos pasos que utiliza el ensamblador para \textit{MIPS32},
y es que este ensamblador presenta \textbf{la misma arquitectura}
que el ensamblador presente por defecto en \textit{JAMS}.

\subsection{Descubrimiento}\label{subsec:descubrimiento}

En este paso el texto del proyecto se \textbf{descompone en sus primitivas},
permitiendo al ensamblador entender los diferentes componentes de cada línea.
Al final de este paso, las etiquetas globales y las etiquetas del archivo
(etiquetas no definidas dentro de una macro) \textbf{son registradas sin
ningún valor asignado}.
Las macros de cada archivo también son registradas.
El identificador de una macro es definido por su nombre concatenado-*
al número de parámetros que necesitan.
Este procedimiento se realiza para dar soporte a la sobrecarga de macros.
En el caso de la macro $print$, su identificador sería $print-1$.

Cabe destacar que, a diferencia del ensamblador para \textit{MIPS32},
las etiquetas pueden hacer referencia a una dirección de memoria
o a una \textbf{equivalencia}.
Estas equivalencias relacionan una etiqueta a una expresión matemática
que a la vez puede usar otras etiquetas para calcular su valor.

\subsection{Expansión}\label{subsec:expansion}

En este paso, las llamadas a macros son invocadas,
insertando el código de la macro en la posición de la llamada.
Este código efectúa el primer paso del ensamblador mientras es añadido.
Al ser insertado justo después de la llamada, el código de la macro
también será expandido.

\subsubsection{Alcance}\label{subsubsec:alcance}

Las etiquetas y macros que están dentro de una macro
\textbf{tienen un alcance diferente al del archivo}.
Si la macro es global, el alcance es considerado hijo del alcance global
y no podrá acceder a las etiquetas del archivo que lo invoca.
Si la macro es local, el alcance es considerado hijo del alcance del archivo.

Cuando un alcance es hijo de otro alcance,
\textbf{el hijo podrá acceder a las etiquetas y macros de su padre}.
El hijo también podrá definir nuevas etiquetas y macros con el mismo
identificador que una etiqueta o macro de su padre.
Aunque este comportamiento está permitido, \textbf{el hijo solo podrá acceder
al elemento que él define}.
Esta funcionalidad es llamada \textbf{ocultamiento o \textit{shadowing}}.

\subsection{Asignación de direcciones}\label{subsec:asignacion-de-direcciones}

Una vez el ensamblador haya expandido las macros,
se asignan las direcciones de todas las instrucciones,
etiquetas y directivas que requieran dirección.
Estas direcciones se asignan de manera secuencial.
Existen directivas que pueden modificar el flujo de la asignación,
como es el caso de la directiva $.bank$ anteriormente mencionada.

\subsection{Asignación de valores}\label{subsec:asignacion-de-valores}

Como paso final, el ensamblador insertará en memoria los valores
que representan las directivas e instrucciones.
Es en este paso donde se resuelven los valores de las equivalencias.

\subsection{Empaquetamiento}\label{subsec:empaquetamiento}

Como paso adicional fuera del ensamblador, los datos resultantes
en los bancos de memoria son empaquetados en un archivo \textit{iNES}.
Este archivo junta la cabecera especificada en la configuración seleccionada
por el usuario, los bancos de memoria especificados como memoria
\textit{ROM} y los gráficos traducidos a tablas de patrones.

\subsection{Características avanzadas}\label{subsec:características-avanzadas}

El ensamblador permite el uso de técnicas avanzadas en
el desarrollo de aplicaciones en lenguaje ensamblador.

\subsubsection{Referencias relativas}\label{subsubsec:referencias-relativas}

Una directiva o instrucción puede \textbf{referenciar a una etiqueta de manera
relativa} con las referencias especiales $+$ y $-$.
La referencia $+$ hace referencia a la etiqueta siguiente.
La referencia $-$ hace referencia a la etiqueta anterior.
Las referencias relativas \textbf{solo pueden hacer referencia
a etiquetas del mismo alcance}.
No pueden hacer referencia a etiquetas de un alcance mayor.

\subsubsection{Macros anidadas}\label{subsubsec:macros-anidadas}

Una macro puede ser definida dentro de otra macro.
Esto es conocido como una \textbf{macro anidada}.
Esta macro solo podrá ser accedida en el alcance de la macro
en la que está declarada.

\subsubsection{Expresiones}\label{subsubsec:expresiones}

Las expresiones son la característica más avanzada
presente en el ensamblador.
Esta característica permite deducir el valor
de una instrucción o directiva mediante una \textbf{expresión matemática}
que puede usar etiquetas como parámetros.

\begin{figure}[h]
    \centering
    \includegraphics[width=0.8\textwidth]{images/nes/nes-expressions}
    \caption{Expresiones con sumas usadas en instrucciones}
    \label{fig:nes-expressions}
\end{figure}

Los usuarios pueden emplear una gran variedad de operaciones
en las expresiones, como son la suma, la resta, la multiplicación,
la división o las operaciones a nivel de bit.
También existen operaciones \textbf{unarias}, como son
la conversión de un número en byte o palabra,
la negación a nivel de bit o la selección del
primer o segundo byte de una palabra.

\subsection{Detalles finales}\label{subsec:detalles-finales}

A diferencia del ensamblador para \textit{MIPS32},
este ensamblador no es muy personalizable.
Esto es debido a que \textit{NES4JAMS} pretende
ser un entorno de desarrollo para videojuegos de la \textit{NES}
que puedan ejecutarse en \textbf{consolas reales}.
\textit{NES4JAMS} da soporte para todas las
\textbf{instrucciones legales} presentes en la consola,
por lo que no es factible dar soporte para instrucciones
de terceros.
    \clearpage{\pagestyle{empty}\cleardoublepage}

    %\chapter{Resultados}\label{ch:resultados}


\section{Resultados relativos al objetivo 1}\label{sec:resultados-relativos-al-objetivo-1}

Los resultados de este objetivo corresponden a la investigación de
tecnologías adecuadas para el desarrollo de la aplicación,
buscando un lenguaje de programación y un \textit{framework} de
desarrollo de aplicaciones de escritorio que permitan crear
aplicaciones modernas, multiplataforma, de gran calidad,
y que permitan cargar código externo a voluntad.

\noindent Se considera que \textbf{se ha realizado una búsqueda adecuada}
del lenguaje de programación y del \textit{framework}, los cuales
terminaron siendo \textit{Java 17} y \textit{JavaFX}.
Con estas tecnologías se ha conseguido desarrollar una aplicación
moderna y multiplataforma, con soporte para componentes y totalmente
personalizable.

\begin{figure}[h]
    \centering
    \includegraphics[width=0.8\textwidth]{images/result/idea-plugin}
    \caption{Desarrollo del componente \textit{NES4JAMS} en \textit{Kotlin}}
    \label{fig:jams-plugin}
\end{figure}

\noindent Gracias al uso de una versión de \textit{Java} moderna,
el desarrollo de la aplicación ha sido \textbf{mucho más rápido} y el resultado
ha sido \textbf{mucho más profesional}, pudiendo implementar diseños y
algoritmos rápidamente y con menores errores.
\textit{JavaFX} ha permitido desarrollar una interfaz de usuario
que se despega del conocido y desfasado formato de las aplicaciones
\textit{Swing}.
Gracias a su fácil desarrollo y su capacidad de usar código
\textit{CSS} para definir el estilo de la aplicación, \textit{JAMS}
puede presumir de tener un aspecto moderno y profesional.

\noindent Una de las ventajas que fue apareciendo durante el
desarrollo con respecto a utilizar \textit{Java} ha sido la
capacidad de poder usar otros lenguajes de programación capaces
de compilar a la \textit{JVM} para el desarrollo de componentes.
El desarrollador podrá escoger de un \textbf{amplio abanico de lenguajes}
de programación para el desarrollo de su componente.
Algunos ejemplos de lenguajes de programación que compilan a la \textit{JVM}
son \textit{Scala}, \textit{Kotlin} o \textit{Groovy}.
Un ejemplo de componente desarrollado en \textit{Kotlin} sería
\textit{NES4JAMS}, el cual es parte del Trabajo de Fin de Grado del Grado en
Diseño y Desarrollo de Videojuegos y cuyo método principal se puede observar
en la figura \ref{fig:jams-plugin}.


\section{Resultados relativos al objetivo 2}\label{sec:resultados-relativos-al-objetivo-2}

Los resultados de este objetivo corresponden a la creación de
un entorno base y un \textit{framework} que permita \textbf{implementar
diferentes entornos y herramientas}, creando así una capa
de abstracción que ayude a los desarrolladores y al propio
\textit{JAMS} a crear herramientas de manera rápida y sencilla.

\noindent Se considera que \textbf{se ha superado con creces}
el objetivo.
Gracias a las diferentes tecnologías que se han desarrollado
para la base, se pueden crear nuevas herramientas en cuestión
de minutos.
Los objetivos definidos en el Trabajo de Fin de Grado del Grado en
Diseño y Desarrollo de Videojuegos y mencionados en esta memoria
se han desarrollado paralelamente a este objetivo, resultando
en una aplicación base \textbf{robusta}, con capacidad de personalizar
una gran cantidad de aspectos del entorno.

\noindent La base también permite a los usuarios más comunes e
inexpertos personalizar el entorno de desarrollo gracias a la
\textbf{extensa configuración} y la capacidad de poder producir
\textbf{paquetes de idiomas y de temas}.
El resultado de estas personalizaciones se puede apreciar
perfectamente en la figura \ref{fig:jams-collage}.

\begin{figure}[h]
    \centering
    \includegraphics[width=0.8\textwidth]{images/result/jams-collage}
    \caption{Diferentes perfiles de personalización de JAMS}
    \label{fig:jams-collage}
\end{figure}

\noindent Por último, destacar el resultado de la \textit{interfaz de usuario}.
El uso de nodos como elemento central de la interfaz puede considerarse
un \textbf{acierto}: son componentes muy fáciles de usar y altamente personalizables.
Estos nodos son independientes entre sí, y muchos de ellos son \textbf{independientes}
de la tecnología que esté empleando el usuario, como es el caso del \textbf{explorador}.
Esto permite que sean herramientas \textbf{altamente reutilizables}, estando en todo
momento a disposición del desarrollador de componentes.


\section{Resultados relativos al objetivo 3}\label{sec:resultados-relativos-al-objetivo-3}

Los resultados de este objetivo corresponden al desarrollo
de un entorno de desarrollo para la arquitectura \textit{MIPS32}
usando la base creada en el objetivo anterior.
El objetivo requiere de la creación de un editor, un ensamblador
y un simulador, además de diferentes herramientas que complementen
a estos tres elementos principales.

\noindent Se considera que este objetivo se ha superado
de manera óptima.
\textit{JAMS} presenta de base un entorno de desarrollo completo
para la arquitectura \textit{MIPS32}.

\noindent El \textbf{editor de texto} está al nivel de los editores de texto
inteligentes que se pueden encontrar en los entornos de desarrollo
actuales, \textbf{ayudando al usuario} en la mayoría de las tareas
relacionadas con desarrollar una aplicación en ensamblador, como se observa
en la figura \ref{fig:mips-editor}.

\begin{figure}[h]
    \centering
    \includegraphics[width=\textwidth]{images/result/mips-editor}
    \caption{Editor de texto proporcionando ayuda al usuario}
    \label{fig:mips-editor}
\end{figure}

\noindent El \textbf{ensamblador} es altamente \textbf{personalizable},
permitiendo que otros componentes puedan aportar nuevas instrucciones y
directivas de manera sencilla.
Este ensamblador también incorpora varias \textbf{características avanzadas}
que el usuario puede usar, como son las \textbf{macros} y las
\textbf{etiquetas relativas}.

\noindent El \textbf{simulador} permite ejecutar código ensamblador
\textbf{MIPS32} en diferentes arquitecturas, siendo la arquitectura
uniciclo la más rápida de todas ellas, llegando a superar los
\textbf{40 millones de ciclos cada segundo}.
El simulador viene altamente equipado con diversas herramientas que
permiten al usuario \textbf{visualizar y modificar} el estado del simulador
de diversas maneras, tal y como se puede observar en la figura \ref{fig:mips-tools}.
Como detalle final, el simulador presenta una estructura basada en
\textbf{hilos}, lo que evita que la aplicación se congele al ejecutar
una aplicación.

\begin{figure}[h]
    \centering
    \includegraphics[width=\textwidth]{images/result/mips-tools}
    \caption{Todas las herramientas proporcionadas por el simulador}
    \label{fig:mips-tools}
\end{figure}
    %\clearpage{\pagestyle{empty}\cleardoublepage}

    %\chapter{Conclusiones}\label{ch:conclusiones}

Los objetivos de este proyecto estaban asociados a la \textbf{creación
de un nuevo entorno de desarrollo} especializado en lenguajes
ensamblador que pudieran usar tanto desarrolladores avanzados
como alumnos.

\noindent \textit{JAMS} es un entorno de desarrollo moderno,
flexible, modular y fácil de usar, donde el usuario puede
crear aplicaciones de una manera rápida y cómoda,
apoyándose en las diferentes herramientas y
características que la aplicación aporta.

\noindent La aplicación viene empaquetada junto con un
editor, un ensamblador y un simulador para la arquitectura
\textit{MIPS32}, permitiendo personalizar la manera en la
que un proyecto es ejecutado.

\noindent Gracias a la buena elección de tecnologías y
\textit{frameworks}, la experiencia proporcionada por
\textit{JAMS} es altamente personalizable, permitiendo
al usuario usar y crear temas e idiomas.

\noindent Aunque \textit{JAMS} goce de una arquitectura
totalmente personalizabler, aún no tiene la opción de cargar componentes.
Esta característica será implementada en el Trabajo de Fin de
Grado del Grado en Diseño y Desarrollo de Videojuegos.

\section{Limitaciones}\label{sec:limitaciones}

\textit{JAMS} presenta las siguientes limitaciones:

\begin{itemize}
    \item \textbf{Velocidad}: \textit{JAMS} es capaz de ejecutar
    una simulación a un máximo de \textbf{40 millones de instrucciones por segundo} en un
    procesador \textit{AMD Ryzen 7 2700X}.
    Esta velocidad, aunque supere con creces a la mayoría de simuladores \textit{MIPS32},
    puede impedir ejecutar aplicaciones que requieran de una gran capacidad de cómputo.
    \item \textbf{Soporte para MIPS32r5}: actualmente, \textit{JAMS} solo da soporte
    a la última revisión de la arquitectura \textit{MIPS32}.
    Muchos usuarios siguen utilizando la revisión anterior, por lo que tendrán que migrar
    sus proyectos antes de poder emplear \textit{JAMS}.
\end{itemize}

\section{Líneas futuras}\label{sec:líneas-futuras}

\textit{JAMS} contará con dos actualizaciones importantes en el
futuro cercano.
La primera actualización ya está en desarrollo, mientras que la
segunda actualización requerirá de nueva tecnología que
el equipo de \textit{JAVA} está desarrollando.
\begin{itemize}
    \item \textbf{Soporte para la arquitectura \textit{MIPS32r5}}:
    actualmente, \textit{JAMS} solo soporta proyectos de la revisión 6
    de la arquitectura \textit{MIPS32}.
    Esta revisión cambia la arquitectura en muchos aspectos con respecto
    a la revisión anterior, lo que fuerza a muchas personas a tener
    que migrar gran parte del código de sus proyectos.
    Añadir soporte a la revisión 5 será una tarea relativamente sencilla,
    sabiendo que la revisión 6 añade más características de las que quita.
    \item \textbf{Uso del \textit{Proyecto Valhalla}:} el
    \textit{Proyecto Valhalla}\cite{PROJECT_VALHALLA}, creado en 2014,
    está desarrollando una de las características más esperadas
    por los desarrolladores \textit{JAVA}: \textbf{paso por valor}.
    Esta característica permitirá optimizar de manera considerable
    muchos aspectos de \textit{JAMS}, como son el editor o el simulador.
    Estas nuevas características empezarán su desarrollo cuando esta
    tecnología esté en fase \textit{preview}.
\end{itemize}

\section{Reflexiones finales}\label{sec:reflexiones-finales}

Este ha sido el proyecto más complejo en el que he trabajo:
\textit{JAMS} abarca un montón de conceptos y tecnologías,
desde la simulación de arquitecturas hasta el despliegue de
aplicaciones automatizado, pasando por el renderizado a tiempo
real, la creación de interfaces y la estructuración de un
proyecto grande.

\noindent \textit{JAMS} ha sido una apuesta, un proyecto
que podría haberse derrumbado rápidamente.
Puede considerarse la consolidación de todos los conocimientos
que he adquirido en los últimos años, tanto fuera como dentro
de la universidad.

\noindent Finalmente, agradecer a todas las personas que me han
estado apoyando en este proyecto, a mi familia, amigos y a mis
tutores Óscar y Luis, ya que gracias a ellos este trabajo
ha sido posible.
    %\clearpage{\pagestyle{empty}\cleardoublepage}

    \backmatter
    \bibliography{main}
    \bibliographystyle{orobles}
    \addcontentsline{toc}{chapter}{Bibliografía}
\end{document}
