\chapter{Entorno de desarrollo \textit{NES}}\label{ch:entorno-de-desarrollo-nes}

En este capítulo se abordará el desarrollo de \textit{NES4JAMS},
un componente que aporta a \textit{JAMS} un entorno de desarrollo
integrado para la creación de videojuegos de la consola
\textit{NES}.
Gracias a la arquitectura de \textit{JAMS}, este componente
aprovechará muchas de las características desarrolladas
para el entorno de desarrollo \textit{MIPS32} presente
por defecto en la aplicación.
Este entorno de desarrollo consistirá de un editor
de texto, un ensamblador y un simulador.


\section{Creación de un proyecto}\label{sec:creacion-de-un-proyecto}

\begin{figure}[h]
    \centering
    \includegraphics[width=0.8\textwidth]{images/nes/nes-project-creation}
    \caption{Creación de un proyecto \textit{NES}}
    \label{fig:nes-project-creation}
\end{figure}

\textit{NES4JAMS} añade un nuevo tipo de proyecto
que incorpora todas las herramientas para el desarrollo
de videojuegos de \textit{NES}.
Los usuarios podrán crear nuevos proyectos de este tipo
desde el menú de creación de proyectos.
Como los proyectos de \textit{NES} están destinados
a una consola muy específica, el usuario solo debe
especificar \textbf{el nombre y la localización del proyecto},
como se puede observar en la figura \ref{fig:nes-project-creation}.

Una vez creado el proyecto, \textit{JAMS} mostrará una ventana
principal muy similar a la de los proyectos \textit{MIPS32}.


\section{Editor}\label{sec:editor}

Los desarrolladores de videojuegos de \textit{NES} trabajan con
dos tipos de archivo principales: los archivos \textit{asm}
para el código y los archivos \textit{pcx} para los gráficos.
\textit{NES4JAMS} permite al usuario editar los dos tipos de archivo
de manera sencilla.

Cuando el usuario desea editar uno de estos archivos,
debe seleccionarlo en la herramienta \textbf{explorador}.
Esta herramienta muestra una representación en forma de árbol
de la estructura del proyecto.
El usuario puede expandir y contraer carpetas, así como crear,
borrar y mover archivos.
Si el usuario usa el doble clic sobre un archivo editable, este se abrirá
en el editor.
Dependiendo del tipo de archivo, el editor será diferente,
como se puede observar en la figura \ref{fig:nes-editor}.

\begin{figure}[h]
    \centering
    \includegraphics[width=0.8\textwidth]{images/nes/nes-editor}
    \caption{Editor de código y de gráficos junto con el explorador}
    \label{fig:nes-editor}
\end{figure}

El menú contextual del explorador presenta varias acciones que
pueden ejecutarse sobre las carpetas y los archivos
del proyecto.
Una de las opciones más particulares es la opción de añadir o eliminar
archivos de código o de gráficos del ensamblador.
Al ser \textit{JAMS} un entorno de desarrollo basado en \textbf{proyectos},
se ha de proporcionar una manera de incluir o excluir archivos del
videojuego resultante.
Con este simple sistema, el usuario podrá elegir qué archivos se debe ensamblar.
Los archivos a ensamblar estarán marcados en \textbf{verde} en el explorador,
y aparecerán en orden en los nodos \textbf{Archivos a ensamblar} y
\textbf{\textit{Sprites} a ensamblar}.
Cabe destacar que, como se verá más adelante, los el orden de ensamblaje
importa, por lo que esta herramienta permite ordenarlos de una manera sencilla.

Una vez el usuario abra un archivo, su editor aparecerá en la herramienta
principal de la sección: \textbf{el visualizador de archivos}.
\textit{NES4JAMS} implementa dos editores nuevos a \textit{JAMS}:
el editor de código \textit{MOS 6502} y el editor de gráficos \textit{PCX}.

\subsection{Editor de código}\label{subsec:editor-de-codigo}

El editor de código usa el sistema de indexación desarrollado
en la capa base de \textit{JAMS}, por lo que este editor también
puede considerarse un \textbf{editor de texto inteligente}:
el editor convierte el texto puro en los componentes ensamblador
representados, pudiendo así aportar ayudas al usuario.
El editor también tiene conocimiento de las referencias y el alcance de
todas las etiquetas y macros, tanto en el propio archivo a editar
como en el resto de archivos a ensamblar.
El editor también incorpora un \textbf{autocompletador}.
Esta herramienta ayuda al usuario cuando escribe código
aportando sugerencias de autocompletación, como se puede
observar en la figura \ref{fig:nes-autocompletion}.

\begin{figure}[h]
    \centering
    \includegraphics[width=0.8\textwidth]{images/nes/nes-autocompletion}
    \caption{Autocompletador ayudando al usuario}
    \label{fig:nes-autocompletion}
\end{figure}

\subsection{Editor de gráficos}\label{subsec:editor-de-graficos}

El formato \textit{PCX} es un formato de imagen desarrollado
en el año 1985 que cuenta con una codificación \textit{run-length}.
Actualmente, el formato ha caído en desuso, pero sus propiedades
lo hacen un \textbf{gran candidato} para almacenar gráficos
de la \textit{NES} antes de ser ensamblados.
Esto se debe a su capacidad de codificar los colores mediante
una \textit{paleta} de una manera sencilla.
Al almacenar los valores de los píxeles como índices y no
como colores, el formato \textit{PCX} se convierte en un
candidato ideal para gráficos dependientes de una paleta
externa, como es el caso de los gráficos de una \textit{NES}.

El editor de archivos \textit{PCX} permite modificar los
gráficos del videojuego de una manera rápida y sencilla.
Este editor está pensado exclusivamente para gráficos
de \textit{NES}, por lo que cada pixel solo puede tomar
cuatro valores diferentes.
El color final se buscará en la paleta seleccionada.
El usuario puede cambiar el color de cada valor de la
paleta utilizando el botón central del ratón sobre
la casilla a cambiar.
Por motivos de accesibilidad, el usuario también puede
cambiar el color empleando el botón principal del
ratón mientras mantiene pulsada la tecla $Ctrl$.

Cabe destacar que, aunque este editor permite editar
archivos \textit{PCX} de cualquier tamaño, la consola
leerá el archivo como si tuviera un ancho de 128 píxeles.
Esto es debido a que las tablas donde se guardan los gráficos
en la consola tienen un tamaño de 16x16 patrones
de 8x8 píxeles cada uno.
Si se ensambla un archivo \textit{PCX} con otro ancho,
lo más probable es que los gráficos se conviertan en
\textbf{ruido}.
Los archivo \textit{PCX} creados por \textit{NES4JAMS}
siempre tienen el tamaño de una tabla de patrones,
es decir, de 128x128 píxeles.

\begin{figure}[h]
    \centering
    \includegraphics[width=0.8\textwidth]{images/nes/nes-graphics-change}
    \caption{\textit{Super Mario Bros.} con los gráficos modificados}
    \label{fig:nes-graphics-change}
\end{figure}

\subsection{Archivos iNES}\label{subsec:archivos-nes}

El formato \textit{iNES} es el formato de archivo más utilizado
para almacenar videojuegos para la consola \textit{NES}.
Este formato comienza con una cabecera con los datos que suelen
estar codificados en el propio cartucho, como puede ser el modo
espejo, el controlador de memoria o la región.
Esta cabecera también contiene el tamaño de la región del
programa (\textit{PGR}) y la región de gráficos (\textit{CHR})
que prosiguen a la cabecera.

\textit{NES4JAMS} es capaz de \textbf{cargar y generar} archivos
\textit{iNES} de manera nativa.
Al abrir un archivo \textit{iNES}, \textit{NES4JAMS} mostrará
el editor mostrado en la figura \ref{fig:nes-ines-editor}.

\begin{figure}[h]
    \centering
    \includegraphics[width=0.8\textwidth]{images/nes/nes-ines-editor}
    \caption{Archivo \textit{iNES} en el editor}
    \label{fig:nes-ines-editor}
\end{figure}

En este editor el usuario podrá ejecutar una simulación
para el videojuego seleccionado, permitiéndole \textbf{modificar}
los parámetros de la cabecera del videojuego a cargar.
El usuario también tiene la opción de \textbf{exportar las tablas
de patrones} del cartucho en un archivo \textit{PCX}.
Una previsualización de estas tablas se puede observar en la parte
derecha del editor.
Por último, el editor da la opción de \textbf{guardar una copia}
del videojuego con los parámetros modificados.

\subsection{Configuraciones}\label{subsec:configuraciones}

\begin{figure}[h]
    \centering
    \includegraphics[width=0.8\textwidth]{images/nes/nes-configurations}
    \caption{Menú de configuraciones}
    \label{fig:nes-configurations}
\end{figure}

Igual que en el entorno de desarrollo para \textit{MIPS32},
el entorno de desarrollo \textit{NES} permite ensamblar
el programa con diferentes \textbf{configuraciones}.
Estas configuraciones definen los \textbf{parámetros de la cabecera}
del archivo \textit{iNES} resultante y los \textbf{bancos de memoria}
presentes en el cartucho.

Los bancos de memoria están definidos por una \textbf{dirección de inicio}
y un \textbf{tamaño} en bytes.
De manera muy similar a las directivas \textit{.text}
y \textit{.data} de \textit{MIPS32}, el desarrollador podrá
alternar entre bancos de memoria usando la directiva \textit{.bank}.

\textit{NES4JAMS} también incorpora una opción que impide a un
banco de memoria ser escrito en el cartucho.
Esta funcionalidad es muy importante para algunos videojuegos
avanzados, ya que les permite utilizar un banco de memoria como
una representación de una \textbf{memoria RAM} adicional
dentro del cartucho.


\section{Ensamblador}\label{sec:ensamblador}

El ensamblador para el lenguaje ensamblador \textit{MOS 6502}
es el ensamblador usado para ensamblar videojuegos para la \textit{NES}.
Este ensamblador soporta características avanzadas empleadas
comúnmente al programar en ensamblador, como las macros,
las etiquetas globales o las referencias relativas.
El ensamblador ensambla el código de un proyecto en \textbf{cuatro pasos}:
descubrimiento, expansión, asignación de direcciones y asignación de valores.
Estos son los mismos pasos que utiliza el ensamblador para \textit{MIPS32},
y es que este ensamblador presenta \textbf{la misma arquitectura}
que el ensamblador presente por defecto en \textit{JAMS}.

\subsection{Descubrimiento}\label{subsec:descubrimiento}

En este paso el texto del proyecto se \textbf{descompone en sus primitivas},
permitiendo al ensamblador entender los diferentes componentes de cada línea.
Al final de este paso, las etiquetas globales y las etiquetas del archivo
(etiquetas no definidas dentro de una macro) \textbf{son registradas sin
ningún valor asignado}.
Las macros de cada archivo también son registradas.
El identificador de una macro es definido por su nombre concatenado-*
al número de parámetros que necesitan.
Este procedimiento se realiza para dar soporte a la sobrecarga de macros.
En el caso de la macro $print$, su identificador sería $print-1$.

Cabe destacar que, a diferencia del ensamblador para \textit{MIPS32},
las etiquetas pueden hacer referencia a una dirección de memoria
o a una \textbf{equivalencia}.
Estas equivalencias relacionan una etiqueta a una expresión matemática
que a la vez puede usar otras etiquetas para calcular su valor.

\subsection{Expansión}\label{subsec:expansion}

En este paso, las llamadas a macros son invocadas,
insertando el código de la macro en la posición de la llamada.
Este código efectúa el primer paso del ensamblador mientras es añadido.
Al ser insertado justo después de la llamada, el código de la macro
también será expandido.

\subsubsection{Alcance}\label{subsubsec:alcance}

Las etiquetas y macros que están dentro de una macro
\textbf{tienen un alcance diferente al del archivo}.
Si la macro es global, el alcance es considerado hijo del alcance global
y no podrá acceder a las etiquetas del archivo que lo invoca.
Si la macro es local, el alcance es considerado hijo del alcance del archivo.

Cuando un alcance es hijo de otro alcance,
\textbf{el hijo podrá acceder a las etiquetas y macros de su padre}.
El hijo también podrá definir nuevas etiquetas y macros con el mismo
identificador que una etiqueta o macro de su padre.
Aunque este comportamiento está permitido, \textbf{el hijo solo podrá acceder
al elemento que él define}.
Esta funcionalidad es llamada \textbf{ocultamiento o \textit{shadowing}}.

\subsection{Asignación de direcciones}\label{subsec:asignacion-de-direcciones}

Una vez el ensamblador haya expandido las macros,
se asignan las direcciones de todas las instrucciones,
etiquetas y directivas que requieran dirección.
Estas direcciones se asignan de manera secuencial.
Existen directivas que pueden modificar el flujo de la asignación,
como es el caso de la directiva $.bank$ anteriormente mencionada.

\subsection{Asignación de valores}\label{subsec:asignacion-de-valores}

Como paso final, el ensamblador insertará en memoria los valores
que representan las directivas e instrucciones.
Es en este paso donde se resuelven los valores de las equivalencias.

\subsection{Empaquetamiento}\label{subsec:empaquetamiento}

Como paso adicional fuera del ensamblador, los datos resultantes
en los bancos de memoria son empaquetados en un archivo \textit{iNES}.
Este archivo junta la cabecera especificada en la configuración seleccionada
por el usuario, los bancos de memoria especificados como memoria
\textit{ROM} y los gráficos traducidos a tablas de patrones.

\subsection{Características avanzadas}\label{subsec:características-avanzadas}

El ensamblador permite el uso de técnicas avanzadas en
el desarrollo de aplicaciones en lenguaje ensamblador.

\subsubsection{Referencias relativas}\label{subsubsec:referencias-relativas}

Una directiva o instrucción puede \textbf{referenciar a una etiqueta de manera
relativa} con las referencias especiales $+$ y $-$.
La referencia $+$ hace referencia a la etiqueta siguiente.
La referencia $-$ hace referencia a la etiqueta anterior.
Las referencias relativas \textbf{solo pueden hacer referencia
a etiquetas del mismo alcance}.
No pueden hacer referencia a etiquetas de un alcance mayor.

\subsubsection{Macros anidadas}\label{subsubsec:macros-anidadas}

Una macro puede ser definida dentro de otra macro.
Esto es conocido como una \textbf{macro anidada}.
Esta macro solo podrá ser accedida en el alcance de la macro
en la que está declarada.

\subsubsection{Expresiones}\label{subsubsec:expresiones}

Las expresiones son la característica más avanzada
presente en el ensamblador.
Esta característica permite deducir el valor
de una instrucción o directiva mediante una \textbf{expresión matemática}
que puede usar etiquetas como parámetros.

\begin{figure}[h]
    \centering
    \includegraphics[width=0.8\textwidth]{images/nes/nes-expressions}
    \caption{Expresiones con sumas usadas en instrucciones}
    \label{fig:nes-expressions}
\end{figure}

Los usuarios pueden emplear una gran variedad de operaciones
en las expresiones, como son la suma, la resta, la multiplicación,
la división o las operaciones a nivel de bit.
También existen operaciones \textbf{unarias}, como son
la conversión de un número en byte o palabra,
la negación a nivel de bit o la selección del
primer o segundo byte de una palabra.

\subsection{Detalles finales}\label{subsec:detalles-finales}

A diferencia del ensamblador para \textit{MIPS32},
este ensamblador no es muy personalizable.
Esto es debido a que \textit{NES4JAMS} pretende
ser un entorno de desarrollo para videojuegos de la \textit{NES}
que puedan ejecutarse en \textbf{consolas reales}.
\textit{NES4JAMS} da soporte para todas las
\textbf{instrucciones legales} presentes en la consola,
por lo que no es factible dar soporte para instrucciones
de terceros.