\chapter{Resultados}\label{ch:resultados}


\section{Resultados relativos al objetivo 1}\label{sec:resultados-relativos-al-objetivo-1}

Los resultados de este objetivo corresponden al desarrollo
de un \textbf{sistema de vinculación de componentes} para el
entorno de desarrollo integrado \textit{JAMS},
permitiendo a otros desarrolladores aportar nuevas funcionalidades
a la aplicación de manera sencilla.

Este objetivo se ha conseguido, ya que el sistema de
vinculación de componentes desarrollado permite
instalar y desinstalar componentes sin que el usuario tenga
que reiniciar la aplicación.
Además, mediante el sistema de dependencias débiles y fuertes,
los componentes pueden \textbf{interactuar entre sí}, añadiendo
más flexibilidad a los desarrolladores.

\begin{figure}[h]
    \centering
    \includegraphics[width=0.8\textwidth]{images/results/jams-uninstall}
    \caption{Desinstalación de un componente}
    \label{fig:jams-uninstall}
\end{figure}

El hecho de que \textit{JAMS} esté escrito en \textit{Java}
ha permitido simplificar mucho el desarrollo de este sistema
de vinculación.
Gracias a las librerías presentes en el \textit{JDK} para cargar
código externo, \textit{JAMS} puede gozar de un sistema de componentes
muy sólido.

El sistema de \textbf{proveedores} desarrollado para los elementos
proporcionados por los componentes también ha resultado ser un éxito.
Este sistema permite que ninguna referencia a ninguna clase de un
componente quede referenciada en \textit{JAMS} cuando dicho componente
se desinstala.
Sin el sistema de proveedores, la desinstalación sería imposible debido
a los errores que se podrían producir.


\section{Resultados relativos al objetivo 2}\label{sec:resultados-relativos-al-objetivo-2}

Los resultados de este objetivo corresponden al desarrollo y mejora de
\textbf{diversas tecnologías} para permitir a los componentes modificar
todos los aspectos de \textit{JAMS} de una manera rápida y sencilla.

Este objetivo también se ha alcanzado.
La tecnología desarrollada más usada por los componentes
y por el propio \textit{JAMS} es el sistema de \textbf{gestores},
que ha permitido gestionar los elementos de la aplicación
de una manera rápida y sencilla, haciendo que los componentes
puedan añadir nuevos elementos de forma transparente.
Este sistema se complementa con los \textbf{eventos},
que permiten recibir información sobre los sucesos
acontecidos en el entorno de desarrollo.
Combinando ambos sistemas, tanto \textit{JAMS} como los componentes
pueden refrescar su información fácilmente cuando se añade o elimina
algún elemento de un gestor.

\begin{figure}[h]
    \centering
    \includegraphics[width=0.8\textwidth]{images/results/jams-tasks}
    \caption{Barra de tareas informando sobre el ensamblaje}
    \label{fig:jams-tasks}
\end{figure}

Otra tecnología muy utilizada han sido las \textbf{tareas},
que permiten ejecutar código asíncrono de una manera muy
sencilla.
Gracias a tener todas las tareas de un proyecto agrupadas
en un ejecutor, se ha podido añadir
un elemento en la barra inferior de la aplicación que muestra
el proceso de todas las tareas del proyecto actual,
tal como se muestra en la figura \ref{fig:jams-tasks}.


\section{Resultados relativos al objetivo 3}\label{sec:resultados-relativos-al-objetivo-3}

Los resultados de este objetivo corresponden al desarrollo
de un componente que añada a \textit{JAMS} un \textbf{entorno de
desarrollo integrado} para la creación de videojuegos de la
consola \textit{NES}, incorporando un editor de código
y de gráficos, un ensamblador y un simulador.

Este objetivo también se considera superado de manera satisfactoria.
El nuevo componente permite desarrollar videojuegos de
la \textit{NES} usando un editor de código muy similar al del
entorno de desarrollo para \textit{MIPS32}, incorporando
herramientas como el \textbf{autocompletador} o el \textbf{inspector de código},
que permite agilizar el desarrollo de los videojuegos.
El componente también presenta un editor de gráficos, que permite
editar los patrones presentes en las tablas del juego.

\begin{figure}[h]
    \centering
    \includegraphics[width=0.8\textwidth]{images/results/nes-editor}
    \caption{Editor de código y de gráficos}
    \label{fig:nes-result-editor}
\end{figure}

El ensamblador, inspirado por el ensamblador presente
en \textit{JAMS} para la arquitectura \textit{MIPS32},
incorpora varias características avanzadas no encontradas
en ensambladores más sencillos, como lo son las referencias
relativas o las macros anidadas.
Este ensamblador escribe el resultado en un archivo \textit{iNES},
lo que permite al desarrollador distribuir su videojuego de manera
sencilla.

Por último, \textit{NES4JAMS} incorpora un simulador de la arquitectura
de la consola \textit{NES} que permite ejecutar la mayoría de los
videojuegos comerciales.
Este simulador permite visualizar el estado del videojuego desde varios
puntos de vista gracias a las herramientas incluidas.
La herramienta más característica del simulador es la herramienta
\textit{PPU}, que presenta de manera muy visual el estado
de la unidad de procesamiento de imágenes,
como se observa en la figura \ref{fig:nes-result-simulator}.

\begin{figure}[h]
    \centering
    \includegraphics[width=0.8\textwidth]{images/results/nes-simulator}
    \caption{Simulador con la herramienta \textit{PPU}}
    \label{fig:nes-result-simulator}
\end{figure}

Todos estos elementos conforman un entorno de desarrollo
\textbf{sólido} para la creación de nuevos videojuegos
para la \textit{NES}.

Por último, cabe destacar que la elección del lenguaje
de programación \textit{Kotlin} para el desarrollo de
este componente ha sido una totalmente adecuada.
Su diseño más conciso, combinado con nuevas características
como el soporte para números sin signo, ha permitido
acelerar el ritmo de desarrollo del componente.
La capacidad de \textit{Kotlin} de permitir ejecutar sus
programas en la \textit{JVM} permite que se puedan
crear componentes para \textit{JAMS} en este lenguaje
de programación de manera \textbf{transparente}, sin
que el desarrollador tenga que ejecutar ningún paso intermedio.
